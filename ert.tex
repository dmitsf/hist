%%%%%%%%%%%%%%%%%%%%%%%%%%%%%%%%%%%%%%%%%%%%%%%%%%%%%%%%%%%%%%%%
%%%                                                          %%%
%%% Преамбула документа --- определение параметров страницы  %%%
%%% без необходимости не изменяйте                           %%%
%%%                                                          %%%
%%%%%%%%%%%%%%%%%%%%%%%%%%%%%%%%%%%%%%%%%%%%%%%%%%%%%%%%%%%%%%%%

\documentclass[12pt]{report}

\usepackage{setspace}
\usepackage{ucs}
\usepackage[utf8x]{inputenc}
\usepackage[english,russian]{babel}
\usepackage{epsfig}
\usepackage{amsmath}
\usepackage{graphics}
\usepackage{comment} %%%
%\renewcommand{\baselinestretch}{2}

\usepackage{ccaption}
% заменяем для рисунков ':' после номера рисунка на '.'
\captiondelim{. } % после точки стоит пробел!
%Этот пакет обладает совместимостью с caption2, достаточно указать
%\usepackage[caption2]{ccaption

%\newenvironment{mylisting}
%{\begin{list}{}{\setlength{\leftmargin}{1em}}\item\scriptsize\bfseries}
%{\end{list}}
%
%\newenvironment{mytinylisting}
%{\begin{list}{}{\setlength{\leftmargin}{1em}}\item\tiny\bfseries}
%{\end{list}}
%
\usepackage{listings}
%

\makeatletter

\renewcommand{\labelenumi}{\theenumi)}

%\pagestyle{myheadings
\pagestyle{plain}

\makeatletter
\makeatother

\renewcommand{\labelenumi}{\theenumi)}

%\textheight=25.5cm %%% 25.5
\textheight=25.5cm
\topskip=-0.4cm  %%% 1
\textwidth=15.5cm
\setlength{\textheight}{245mm}
%\oddsidemargin=-0.46cm
\voffset=-20mm
\hoffset=-10mm


\sloppy

\renewcommand{\baselinestretch}{1.1}

%%%%%%%%%%%%%%%%%%%%%%%%%%%%%
%%%                       %%%
%%% Начало тела документа %%%

\begin{document}

%%%%%%%%%%%%%%%%%%%%%%%%%%%%%%%
%%%                         %%%
%%% Начало титульного листа %%%

\thispagestyle{empty}
\begin{center}
%\Large
%Федеральное агентство по образованию\\
%Правительство Российской Федерации \\
Федеральное государственное \\
автономное образовательное учреждение \\
высшего профессионального образования\\

\Large Национальный исследовательский университет\\
\Large <<Высшая школа экономики>>\\


%\large Отделение программной инженерии\\

\end{center}

\vbox{\vspace{2cm}}

\hangindent=8cm
\hangafter=-25
%\noindent Кафедра \\Управления разработкой программного обеспечения


\vbox{\vspace{10mm}}\noindent %
\noindent
%\hfillФролов Дмитрий Сергеевич\\


%\vbox{\vspace{2cm}}

\begin{center}
\large РЕФЕРАТ \\
%\large для поступления в аспирантуру по направлению\\
%\large 09.06.01 - <<Информатика и вычислительная техника>>\\
\large по истории технических наук\\
%\large 05.13.17 - <<Теоретические основы информатики>>\\

%(промежуточный) \\ \medskip
\large на тему:
%%% Название работы %%%

\huge {\sc
Понятие вероятности в науке \\
и ее применение для решения практических задач
}\\

\vbox{\vspace{6mm}} \noindent

%\large Направление 010400 - Программная инженерия \\

\end{center}

%\vspace{\fill}

\hangindent=8cm
\hangafter=-25
%%% Ваши данные (при сдаче отчёта подпишите его, поставив подпись под своей
%%% фамилией) %%%
\vbox{\vspace{10mm}}\noindent %
%\noindent
Выполнил: аспирант\\
Д.~С.~Фролов \\
%%% Ваш руководитель (ставит оценку и расписывается под фамилией) %%%
%\vbox{\vspace{10mm}}\noindent
%Научный руководитель: \\
%профессор, д. т. н\\
%Б.~Г.~Миркин\\
\vbox{\vspace{10mm}}\noindent
%%% Дата представления и общая оценка за отчёт и сроки
%Представлена ``\rule{0.8cm}{0.4pt}''\rule{2cm}{0.4pt} 2011~г.\\
%Представлен ``\rule{0.8cm}{0.4pt}''\rule{2cm}{0.4pt} 2012~г.\\
\vbox{\vspace{10mm}}\noindent
%Оценка оформления и сроков\\
%представления курсовой работы:\\
%Оценка оформления и сроков\\
%представления:\\
\vbox{\vspace{10mm}}\noindent
%Оценка публичной защиты работы:\\
%Оценка публичной защиты:\\                 %%%%%
\vbox{\vspace{10mm}}\noindent
%Оценки промежуточного отчета:\\
%Оценка научного руководителя:\\            %%%%%
%Оформление и срок представления:\\         
%Оценка оформления\\
%и сроков представления отчета:\\
\vbox{\vspace{10mm}}\noindent
%Итоговая оценка:\\                         %%%%%
\vbox{\vspace{10mm}}\noindent
%Отв. преподаватель:\\                      %%%%%
\vspace{\fill}

\begin{center}
\Large
Москва --- 2014
\end{center}

%%% Конец титульного листа  %%%
%%%                         %%%
%%%%%%%%%%%%%%%%%%%%%%%%%%%%%%%

%\textheight=24.5cm
\textheight=23.0cm

%%%%%%%%%%%%%%%%%%%%%%%%%%%%%%%%
%%%                          %%%
%%% Содержание               %%%

\newpage

\onehalfspacing 


\tableofcontents
\thispagestyle{empty}

%%% Содержание              %%%
%%%                         %%%
%%%%%%%%%%%%%%%%%%%%%%%%%%%%%%%


%%%%%%%%%%%%%%%%%%%%%%%%%%%%%%%%
%%%                          %%%
%%% Введение                 %%%

%%% В введении Вы должны описать предметную область, с которой связана %%%
%%% Ваша работа, показать её актуальность, вкратце определить цель     %%%
%%% исследования/разработки					       %%%

\newpage

\chapter*{Введение}
\addcontentsline{toc}{chapter}{Введение}


\section{Введение}

С развитием вычислительной техники появились задачи автоматической обработки текстов, созданных человеком и написанных на
естественном языке. В силу большой разнообразности естественных языков, огромных словарей и сложности языковых конструкций эти проблемы с трудом поддаются решению.

Одним из направлений в обработке естественного языка является работа с множествами текстовых документов. Сюда можно отнести задачи рубрикации документов, задачи поиска документов по ключевым словам, определение степени близости текстов и другие проблемы.

\section{Постановка задачи}

Одним из факторов, влияющих на сложность задач, связанных с обработкой документов на естественном языке, является количество документов, с которыми предстоит работать, так называемый размер коллекции. Но это не всегда и не только значит, что чем крупнее коллекция, тем сложнее с ней работать. Безусловно, размер коллекции влияет на требования к вычислительным ресурсам, однако также надо понимать, что необходимо правильно выбирать алгоритмы для решения той или иной задачи. 

\subsection{Предварительная обработка документов}


\section{Выводы}

Эксперименты показали, что использование распределенных алгоритмов существенно сокращает время, затрачиваемое на обработку коллекции. Применение предварительного индексирования коллекции по вспомогательным признакам сокращает время выполнения поискового запроса в методе на основе АСД, однако даже с учетом этой модификации этот метод все же несколько уступает в скорости выполнения запроса методам, основанным на LDA и PLSI. 

При сравнении качества поиска метод на основе АСД показал существенное преимущество на <<пользовательских>> запросах.


%%%%%%%%%%%%%%%%%%%%%%%%%%%%%%%%%%%%%%%%%%%%%%%%%%%%%%%%%%%%%%%%%%%%%%%%%%%%%%%%%%%%%%%

\chapter*{Заключение}
\addcontentsline{toc}{chapter}{Заключение}


Главным же направлением дальнейших исследований видится развитие и усовершенствование алгоритмов работы с коллекциями документов на основе метода аннотированных суффиксных деревьев.


\newpage

\renewcommand{\bibname}{Библиографический список использованной литературы}
\begin{thebibliography}{99}
%\thispagestyle{empty}
\vspace{5mm}
\addcontentsline{toc}{chapter}{Библиографический список использованной литературы}

\bibitem{pmasd}
{\it Горяинова, Е. Р., Панков А. Р., Платонов Е. Н. Прикладные методы анализа статистических данных} / Е. Р. Горяинова., А. Р. Панков, Е. Н. Платонов -- М.: Издательский дом Высшей школы экономики, 2012. -- 310 с.

\bibitem{feller}
{\it Феллер, В. Введение в теорию вероятностей и ее приложения. Т.2.} /
В. Феллер; пер. с англ. под ред. Ю. В. Прохорова. -- М.:
Книжный дом <<ЛИБРОКОМ>>, 2010. -- 752 с.

\bibitem{mstat}
{\it Боровков, А. А. Математическая статистика} / А. А. Боровков. -- СПб.: Лань, 2010. -- 704 с.

\bibitem{manning}
{\it Маннинг К. Д., Рагхаван П., Шютце Х. Введение в информационный поиск} / Маннинг К. Д., Рагхаван П., Шютце Х. -- М.: Вильямс, 2011. -- 680 c.

\bibitem{mp}
{\it Шурыгин, А. М. Математические методы прогнозирования} / А. М. Шурыгин. -- М.: Телеком, 2009. -- 180 с.

\bibitem{pms}
{\it Кобзарь, А. И. Прикладная математическая статистика для инженеров и
  научных работников} / А. И. Кобзарь. -- М.: Физматлит, 2006. -- 816 с.

\bibitem{korsh}
{\it Коршунов А, Гомзин А. Тематическое моделирование текстов на естественном языке} // Труды ИСП РАН . 2012. №. С.215-244.

\bibitem{cluster}
{\it Миркин, Б. Г. Методы кластер-анализа для поддержки принятия решений: обзор} / Б. Г. Миркин. -- М.: Издательский дом Национального исследовательского университета <<Высшая школа экономики>>, 2011 -- 84 с.

\bibitem{ch1}
{\it Черняк Е. Л., Миркин Б. Г. Использование ресурсов Интернета для построения таксономии} // В кн.: Доклады всероссийской научной конференции АИСТ 2013 / Отв. ред.: Е. Л. Черняк; науч. ред.: Д. И. Игнатов, М. Ю. Хачай, О. Баринова. М. : Национальный открытый университет <<ИНТУИТ>>, 2013. С. 36-48. 

\bibitem{ch2}
{\it Миркин Б. Г., Черняк Е. Л., Чугунова О. Н. Метод аннотированного суффиксного дерева для оценки степени вхождения строк в текстовые документы }  // Бизнес-информатика. 2012. Т. 3. № 21. С. 31-41. 

\bibitem{vokov2}
{\it  Воронцов К. В., Потапенко А. А. Модификации EM-алгоритма для вероятностного тематического моделирования } // Машинное обучение и анализ данных. -- 2013. -- T. 1, № 6. -- С. 657–686. 


\bibitem{vokov3}
{\it  Иванов М. Н., Воронцов К. В. Применение монотонного классификатора ближайшего соседа в задаче категоризации текстов} // Интеллектуализация обработки информации (ИОИ-2012): Докл.-- Москва: Торус Пресс, 2012. С. 621-624. 

\bibitem{pub1}
{\it М.С. Агеев. Методы автоматической рубрикации текстов, основанных на машинном обучении и знаниях экспертов} / Диссертация на соискание ученой степени к.ф.-м.н. -- М.: МГУ, 2004 

\bibitem{beloz}
{\it Гиляревский Р. С., Шапкин А. В., Белоозеров В. Н. Рубрикатор как инструмент
информационной навигации} -- СПб.: Профессия, 2008. -- 352 с.

\bibitem{dubov}
{\it Дубов М. С., Черняк Е. Л. Аннотированные суффиксные деревья: особенности реализации} // В кн.: Доклады всероссийской научной конференции АИСТ-2013 / Отв. ред.: Е. Л. Черняк; науч. ред.: Д. И. Игнатов, М. Ю. Хачай, О. Баринова. М. : Национальный открытый университет <<ИНТУИТ>>, 2013. С. 49-57.


\bibitem{vokov}
{\it  Ресурсы сайта machinelearning.ru } [Электронный ресурс]. 2012 --. -- Режим доступа: {\it
  http://www.machinelearning.ru/}, свободный. -- Загл. с экрана.

\bibitem{text1}
{\it Турдаков Д. и др. Texterra: инфраструктура для анализа текстов } // ТРУДЫ ИНСТИТУТА СИСТЕМНОГО ПРОГРАММИРОВАНИЯ РАН. -- 2014. -- Т. 26. -- №. 1.

\bibitem{text2}
{\it Велихов П. Е. Использование открытых баз данных как семантических словарей для автоматической обработки текстов: система Texterra} // Информационные процессы. – 2012. – Т. 12. – №. 3. – С. 320-329.

\bibitem{ozon}
{\it OZON.ru -- Каталог в XML } [Electronic resource]. 2009 --. -- Режим доступа: {\it
   http://www.ozon.ru/context/partner\_xml/ }, свободный. -- Загл. с экрана.

\bibitem{segalovich}
{\it  Сегалович И. Как работают поисковые системы} // Мир Internet. -- 2002. -- Т. 10.  

\bibitem{ageev}
{\it Агеев М., Кураленок И., Некрестьянов И. Официальные метрики РОМИП-2004 } // Труды второго российского семинара по оценке методов информационного поиска. Под ред. ИС Некрестьянова-Санкт-Петербург: НИИ Химии СПбГУ. -- 2004. -- С. 142-150.

\bibitem{mirkin}
{\it Mirkin B. G., Core Concepts of Data Analysis} / B. G. Mirkin. -- Springer, 2012. -- 416 p.
 
\bibitem{gensim}
{\it  gensim: Topic modelling for humans } [Electronic resource]. 2009 --. -- Режим доступа: {\it
  http://radimrehurek.com/gensim/ }, свободный. -- Загл. с экрана.

\bibitem{thomas}
{\it  T.K. Moon. The expectation-maximization algorithm } // IEEE Signal Processing Mag., vol. 13, pp. 47–60, Nov. 1996

\bibitem{hofman}
{\it Thomas Hofmann. Probabilistic Latent Semantic Analysis } // UAI 1999: 289-296. 

\bibitem{blei}
{\it D. Blei, A. Ng, M. Jordan. Latent Dirichlet allocation } // Journal of Machine Learning Research, 3: 993–1022, January 2003  

\bibitem{gibbs}
{\it Gregor Heinrich. Parameter estimation for text analysis} // Technical report, Fraunhofer IGD, 2005

\bibitem{gibbs}
{\it Vorontsov K. V., Potapenko A. A. Tutorial on Probabilistic Topic Modeling: Additive Regularization for Stochastic Matrix Factorization } // AIST'2014, Analysis of Images, Social networks and Texts. -- Lecture Notes in Computer Science (LNCS), Springer Verlag-Germany, 2014 (to appear). 

\bibitem{bag}
{\it Steven Bird, Ewan Klein, Edward Loper. Natural Language Processing with Python } / Steven Bird, Ewan Klein, Edward Loper -- OReilly Media, 2009 --. -- 504 p.

\bibitem{s1}
{\it Strehl A., Ghosh J., Mooney R. Impact of similarity measures on web-page clustering} // Workshop on Artificial Intelligence for Web Search (AAAI 2000). – 2000. – С. 58-64.

\bibitem{s2}
{\it Wang Y., Kitsuregawa M. Evaluating contents-link coupled web page clustering for web search results} // Proceedings of the eleventh international conference on Information and knowledge management. – ACM, 2002. – С. 499-506.

\bibitem{s3}
{\it Li T., Chen Y. Web Page Clustering Based on Searching Keywords} // Intelligent Computation Technology and Automation (ICICTA), 2010 International Conference on. – IEEE, 2010. – Т. 3. – С. 1163-1166.

\bibitem{s4}
{\it Langville A. N., Meyer C. D. Google PageRank and beyond: The science of search engine rankings} / -- Princeton University Press, 2011.

\bibitem{s5}
{\it Croft W. B., Metzler D., Strohman T. Search engines: Information retrieval in practice} -- Reading: Addison-Wesley, 2010. -- С. 283.

\bibitem{s6}
{\it Natarajan J. et al. Full-Text Search} // Pro T-SQL 2012 Programmer’s Guide. -- Apress, 2012. -- С. 287-315.

\bibitem{s7}
{\it  Pitts W. M. Full text search capabilities integrated into distributed file systems-Incrementally Indexing Files} : заяв. пат. 13/959,534 США. – 2013.

\bibitem{celery}
{\it  Celery: Distributed Task Queue } [Electronic resource]. 2013 --. -- Режим доступа: {\it
  http://www.celeryproject.org/ }, свободный. -- Загл. с экрана.

\bibitem{mongodb}
{\it  MongoDB  } [Electronic resource]. 2013 --. -- Режим доступа: {\it
  http://www.mongodb.org/ }, свободный. -- Загл. с экрана.

%http://romip.ru/docs/romip_metrics.pdf

\bibitem{vrf}
{\it  C. J. van Rijsbergen. Information Retrieval } / Butterworth's and Co., London, U.K., 2 edition, 1979.


\bibitem{vvv}
{\it  Buckley C., Voorhees E. Evaluating evaluation measure stability. } /  In Proc. of the SIGIR'00, pp. 33-40, 2000. 


\bibitem{chao}
{\it Chao P., Bin W., Chao D. Design and Implementation of Parallel Term Contribution Algorithm Based on Mapreduce Model} // Open Cirrus Summit (OCS), 2012 Seventh. -- IEEE, 2012. -- С. 43-47.

\bibitem{data}
{\it Lin J., Dyer C. Data-intensive text processing with MapReduce} // Synthesis Lectures on Human Language Technologies. -- 2010. -- Т. 3. -- №. 1. -- С. 1-177.

\bibitem{dist}
{\it Zhou B. et al. A distributed text mining system for online web textual data analysis} // Cyber-Enabled Distributed Computing and Knowledge Discovery (CyberC), 2010 International Conference on. -- IEEE, 2010. -- С. 1-4.

\bibitem{kauf}
{\it Hwang K., Dongarra J., Fox G. C. Distributed and cloud computing: From parallel processing to the internet of things.} -- Morgan Kaufmann, 2013.

\bibitem{pattt}
{\it Altevogt P., Nitzsche R. Method of generating a distributed text index for parallel query processing }: пат. 7966332 США. -- 2011.

\bibitem{nltk}
{\it Bird S. NLTK: the natural language toolkit } // Proceedings of the COLING/ACL on Interactive presentation sessions. -- Association for Computational Linguistics, 2006. -- С. 69-72.

\bibitem{ldas}
{\it Wei X., Croft W. B. LDA-based document models for ad-hoc retrieval} // Proceedings of the 29th annual international ACM SIGIR conference on Research and development in information retrieval. -- ACM, 2006. -- С. 178-185.


\end{thebibliography}

%%%                          %%%
%%%%%%%%%%%%%%%%%%%%%%%%%%%%%%%%


\end{document}

%%% Конец тела документа    %%%
%%%                         %%%
%%%%%%%%%%%%%%%%%%%%%%%%%%%%%%%

